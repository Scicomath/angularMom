% !Mode:: "TeX:UTF-8:Hard"
\documentclass{beamer}
\usepackage[UTF8,noindent]{ctexcap}
\usetheme{Madrid}
\usefonttheme[onlymath]{serif}

\usepackage{bm}
\usepackage{physics}

\title{角动量算符对易关系}
\author{艾鑫}
\institute[三峡大学]{三峡大学\hspace{1em}理学院}

\date{\today}

\begin{document}
\maketitle

\section{角动量定义}

\begin{frame}{经典力学中的角动量}
在经典力学中, 角动量的定义为
\begin{equation}
\bm{L} = \bm{r} \times \bm{p}
\end{equation}
展开为
\begin{equation}
L_x = yp_z - zp_y, \quad L_y = zp_x - xp_z, \quad L_z = xp_y - yp_x
\end{equation}

在量子力学中只需将动量换成动量算符, 即为轨道角动量的定义
\end{frame}

\begin{frame}{角动量的一般定义}
角动量的一般定义是通过对易关系定义的
\begin{equation}
\bm{J} \times \bm{J} = i \hbar \bm{J}
\end{equation}
展开为
\begin{equation}
[J_x, J_y] = i\hbar J_z , \quad [J_y, J_z] = i \hbar J_x , \quad [J_z, J_x] = i\hbar J_y
\end{equation}
\end{frame}

\begin{frame}{角动量平方算符}
由于$J_x, J_y, J_z$相互之间不对易, 因此没有共同的本征态. 而角动量平方算符
\begin{equation}
J^2 = J_x^2 + J_y^2 + J_z^2
\end{equation}
却与$J_x, J_y, J_z$都对易,即
\begin{equation}
[J^2,J_x] = 0, \quad [J^2, J_y] = 0, \quad [J^2, J_z] = 0
\end{equation}
我们可以找到$J^2$和$J_z$的共同的本征态$\ket{\psi}$:
\begin{equation}
J^2 \ket{\psi} = \lambda \ket{\psi}, J_z \ket{\psi} = \mu \ket{\psi}.
\end{equation}
\end{frame}

\section{升降算符}

\begin{frame}{升降算符}
定义升降算符
\begin{equation}
J_{\pm} \equiv J_x \pm i J_y
\end{equation}
升降算符有下列性质:
\begin{itemize}
\item ${J_\pm}^\dag = J_\mp$
\item $\comm{J_+}{J_-} = 2 \hbar J_z$
\item $\comm{J_\pm}{J^2} = 0$
\item $\comm{J_\pm}{J_z} = \mp \hbar J_\pm$
\item $J_\pm J_\mp = J^2 - J_z^2 \pm \hbar J_z$
\end{itemize}
\end{frame}

\begin{frame}{升降算符}
如果$\ket{\psi}$是$J^2$和$J_z$的共同本征态, 那么$L_\pm \ket{\psi}$也是它们的本征态.
\begin{gather}
J^2(J_\pm \ket{\psi}) = J_\pm (J^2 \ket{\psi}) = J_\pm  (\lambda \ket{\psi}) = \lambda (J_\pm \ket{\psi}) \\
\begin{split}
J_z (L_\pm \ket{\psi}) &= (J_z J_\pm - J_\pm J_z) \ket{\psi} + J_\pm J_z \ket{\psi} \\
  &= \pm \hbar J_\pm \ket{\psi} + J_\pm (\mu \ket{\psi}) \\
  &= (\mu \pm \hbar) (J_\pm \ket{\psi})
\end{split}
\end{gather}
由上式可知, $J^2$的本征值不变, $J_z$的本征值加(减)一个$\hbar$
\end{frame}

\begin{frame}{升降算符}
对于给定的$\lambda$,不断的施加一个升算符, $J_z$的本征值将不断增加. 但是增加到一程度就会到顶,我们令这个态为$\ket{\psi_t}$, 有
\begin{equation}
J_+ \ket{\psi_t} = 0
\end{equation}
设此时$J_z$的本征值为$\hbar j$:
\begin{equation}
J_z \ket{\psi_t} = \hbar j \ket{\psi_t}, \quad J^2 \ket{\psi_t} = \lambda \ket{\psi_t}
\end{equation}

\end{frame}


\end{document}
