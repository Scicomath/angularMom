\documentclass{beamer}
\usepackage[UTF8,noindent]{ctexcap}
\usetheme{Madrid}
\usefonttheme[onlymath]{serif}

\usepackage{bm}
\usepackage{physics}

\title{角动量算符对易关系}
\author{艾鑫}
\institute[三峡大学]{三峡大学\hspace{1em}理学院}

\date{\today}

\begin{document}
\maketitle

\section{角动量定义}

\begin{frame}{经典力学中的角动量}
在经典力学中, 角动量的定义为
\begin{equation}
\bm{L} = \bm{r} \times \bm{p}
\end{equation}
展开为
\begin{equation}
L_x = yp_z - zp_y, \quad L_y = zp_x - xp_z, \quad L_z = xp_y - yp_x
\end{equation}

在量子力学中只需将动量换成动量算符, 即为轨道角动量的定义
\end{frame}

\begin{frame}{角动量的一般定义}
角动量的一般定义是通过对易关系定义的
\begin{equation}
\bm{J} \times \bm{J} = i \hbar \bm{J}
\end{equation}
展开为
\begin{equation}
[J_x, J_y] = i\hbar J_z , \quad [J_y, J_z] = i \hbar J_x , \quad [J_z, J_x] = i\hbar J_y
\end{equation}
\end{frame}

\begin{frame}{角动量平方算符}
由于$J_x, J_y, J_z$相互之间不对易, 因此没有共同的本征态. 而角动量平方算符
\begin{equation}
J^2 = J_x^2 + J_y^2 + J_z^2
\end{equation}
却与$J_x, J_y, J_z$都对易,即
\begin{equation}
[J^2,J_x] = 0, \quad [J^2, J_y] = 0, \quad [J^2, J_z] = 0
\end{equation}
我们可以找到$J^2$和$J_z$的共同的本征态$\ket{\psi}$:
\begin{equation}
J^2 \ket{\psi} = \lambda \ket{\psi}, J_z \ket{\psi} = \mu \ket{\psi}.
\end{equation}
\end{frame}

\section{升降算符}

\begin{frame}{引入升降算符}
定义升降算符
\begin{equation}
J_{\pm} \equiv J_x \pm i J_y
\end{equation}
\end{frame}

\end{document}
