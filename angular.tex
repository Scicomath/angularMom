% !Mode:: "TeX:UTF-8:Hard"
\documentclass{beamer}
\usepackage[UTF8,noindent]{ctexcap}
\usetheme{Madrid}
\usefonttheme[onlymath]{serif}

\usepackage{bm}
\usepackage{physics}

\title{利用对易关系求角动量算符的矩阵元}
\author{艾鑫}
\institute[三峡大学]{三峡大学\hspace{1em}理学院}

\date{\today}

\begin{document}
\maketitle

\section{角动量定义}

\begin{frame}{目的与方法}
\begin{description}
  \item[目的] 求$J_x,J_y,J_z$和$J^2$算符在$J^2,J_z$共同表象的矩阵元
  \item[方法] 通过角动量算符的对易关系来讨论其本征值问题
\end{description}
\end{frame}

\begin{frame}{经典力学中的角动量}
在经典力学中, 角动量的定义为
\begin{equation}
\bm{L} = \bm{r} \times \bm{p}
\end{equation}
展开为
\begin{equation}
L_x = yp_z - zp_y, \quad L_y = zp_x - xp_z, \quad L_z = xp_y - yp_x
\end{equation}

在量子力学中只需将动量换成动量算符, 即为轨道角动量的定义
\end{frame}

\begin{frame}{角动量的一般定义}
角动量的一般定义是通过对易关系定义的
\begin{equation}
\bm{J} \times \bm{J} = i \hbar \bm{J}
\end{equation}
展开为
\begin{equation}
[J_x, J_y] = i\hbar J_z , \quad [J_y, J_z] = i \hbar J_x , \quad [J_z, J_x] = i\hbar J_y
\end{equation}
\end{frame}

\begin{frame}{角动量平方算符}
由于$J_x, J_y, J_z$相互之间不对易, 因此没有共同的本征态. 而角动量平方算符
\begin{equation}
J^2 = J_x^2 + J_y^2 + J_z^2
\end{equation}
却与$J_x, J_y, J_z$都对易,即
\begin{equation}
[J^2,J_x] = 0, \quad [J^2, J_y] = 0, \quad [J^2, J_z] = 0
\end{equation}
我们可以找到$J^2$和$J_z$的共同的本征态$\ket{\psi}$:
\begin{equation}
J^2 \ket{\psi} = \lambda \ket{\psi}, J_z \ket{\psi} = \mu \ket{\psi}.
\end{equation}
\end{frame}

\section{升降算符}

\begin{frame}{升降算符}
定义升降算符
\begin{equation}
J_{\pm} \equiv J_x \pm i J_y
\end{equation}
升降算符有下列性质:
\begin{itemize}
\item ${J_\pm}^\dag = J_\mp$
\item $\comm{J_+}{J_-} = 2 \hbar J_z$
\item $\comm{J_\pm}{J^2} = 0$
\item $\comm{J_\pm}{J_z} = \mp \hbar J_\pm$
\item $J_\pm J_\mp = J^2 - J_z^2 \pm \hbar J_z$
\end{itemize}
\end{frame}

\begin{frame}{升降算符}
如果$\ket{\psi}$是$J^2$和$J_z$的共同本征态, 那么$L_\pm \ket{\psi}$也是它们的本征态.
\begin{gather}
J^2(J_\pm \ket{\psi}) = J_\pm (J^2 \ket{\psi}) = J_\pm  (\lambda \ket{\psi}) = \lambda (J_\pm \ket{\psi}) \\
\begin{split}
J_z (L_\pm \ket{\psi}) &= (J_z J_\pm - J_\pm J_z) \ket{\psi} + J_\pm J_z \ket{\psi} \\
  &= \pm \hbar J_\pm \ket{\psi} + J_\pm (\mu \ket{\psi}) \\
  &= (\mu \pm \hbar) (J_\pm \ket{\psi})
\end{split}
\end{gather}
由上式可知, $J^2$的本征值不变, $J_z$的本征值加(减)一个$\hbar$
\end{frame}

\begin{frame}{升降算符}
对于给定的$\lambda$,不断的施加一个升算符, $J_z$的本征值将不断增加. 但是增加到一程度就会到顶,我们令这个态为$\ket{\psi_t}$, 有
\begin{equation}
J_+ \ket{\psi_t} = 0
\end{equation}
设此时$J_z$的本征值为$\hbar j$:
\begin{equation}
J_z \ket{\psi_t} = \hbar j \ket{\psi_t}, \quad J^2 \ket{\psi_t} = \lambda \ket{\psi_t}
\end{equation}

\end{frame}

\begin{frame}{升降算符}
现在有
\begin{equation}
\begin{split}\label{JpmJmp}
J_\pm J_\mp &= (J_x \pm i J_y)(J_x \mp i J_y) = J_x^2 + J_y^2 \mp i (J_x J_y - J_y J_x) \\
 &= J^2 - J_z^2 \mp i (i\hbar J_z)
\end{split}
\end{equation}
移项后有
\begin{equation}
J^2 = J_\pm J_\mp + J_z^2 \mp \hbar J_z
\end{equation}
因此有
\begin{equation}
\begin{split}
J^2 \ket{\psi_t} &= (J_- J_+ + J_z^2 + \hbar J_z) \ket{\psi_t} \\
 &= (0 + \hbar^2 j^2 + \hbar^2 j) \ket{\psi_t} \\
 &= \hbar^2 j(j + 1) \ket{\psi_t}
\end{split}
\end{equation}
\end{frame}

\begin{frame}{升降算符}
由本征值方程$J^2 \ket{\psi_t} = \lambda \ket{\psi_t}$得
\begin{equation}
\lambda = \hbar^2 j(j + 1)
\end{equation}
这告诉了我们$J^2$的特征值$\lambda$与$J_z$的最大特征值$j\hbar$的关系
\end{frame}

\begin{frame}{降算符}
同样的道理,对于降算符也有对应的最低的态$\ket{\psi_b}$有
\begin{equation}
J_- \ket{\psi_b} = 0
\end{equation}
类似地,我们令$J_z$对应的本征值为$\hbar \bar{j}$:
\begin{equation}
J_z \ket{\psi_b} = \hbar \bar{j} \ket{\psi_b}, \quad J^2 \ket{\psi_b} = \lambda \ket{\psi_b}
\end{equation}
同样利用\eqref{JpmJmp},我们有
\begin{equation}
\begin{split}
J^2 \ket{\psi_b} &= (J_+ J_- + J_z^2 - \hbar J_z) \ket{\psi_b} \\
 &= (0 + \hbar^2 \bar{j}^2 - \hbar^2 \bar{j}) \ket{\psi_b} \\
 &= \hbar^2 \bar{j}(\bar{j} - 1) \ket{\psi_b}
\end{split}
\end{equation}
\end{frame}

\begin{frame}
比较$\lambda = \hbar^2 j (j + 1)$和$\lambda = \hbar^2 \bar{j}(\bar{j} - 1)$
有
\begin{equation}
j(j+1) = \hbar{j} (\bar{j} + 1)
\end{equation}
要么$\bar{j} = j + 1$, 要么$\bar{j} = -j$. 前面一种情况是不合理的, 故
\begin{equation}
\bar{j} = -j
\end{equation}
\end{frame}

\begin{frame}
设$J_z$的特征值为$m\hbar$, 其中
\begin{equation}
m = -j, -j+1, \cdots, j-1, j
\end{equation}
因此, $j = -j + N$, $N$为某个整数, 所以$j = N/2$,即$j$要么是整数,要么是半整
数:
\begin{equation}
j = 0, 1/2, 1, 3/2, \cdots
\end{equation}
因此我们可以用$j,m$来刻画本征态:
\begin{equation}
J^2 \ket{j,m} = \hbar^2 j (j + 1) \ket{j,m}, \quad J_z \ket{j,m} = \hbar m \ket{j,m}
\end{equation}
\end{frame}

\begin{frame}{矩阵元}
以$\ket{j,m}$为基矢的表象为$J^2$和$J_z$的共同表象. 在此表象中, $J^2$和$J_z$
是对角矩阵, 矩阵元分别为
\begin{gather}
\langle j', m' | J^2 | j,m \rangle = j(j+1)\hbar^2 \delta_{j,j'} \delta_{m,m'} \\
\langle j', m' | J_z | j,m \rangle = m \hbar \delta_{j,j'} \delta_{m,m'}
\end{gather}

下面我们来求$J_\pm$的矩阵元. 设
\begin{equation}
J_\pm \ket{j,m} = C_\pm (j,m) \hbar \ket{j,m \pm 1}
\end{equation}
上式中$C_\pm (j,m)$为待定系数.下面来求$C_\pm (j,m)$. 取上式的厄米共轭,有
\begin{gather}
\bra{j,m}{J_\pm}^\dag = \bra{j,m} J_\mp = \bra{j,m \pm 1} {C_\pm}^* \hbar \\
\langle j,m | J_\mp J_\pm | j,m \rangle = {C_\pm}^*(j,m) C_\pm (j,m) \hbar^2 \braket{j,m\pm 1}{j,m\pm 1}
\end{gather}
\end{frame}

\begin{frame}{矩阵元}
利用归一化条件$\braket{j,m\pm 1}{j,m\pm 1} = 1$以及$J_\mp J_\pm = J^2 -
J_z^2 \mp \hbar J_z$有
\begin{gather}
\begin{split}
|C_\pm (j,m)|^2 \hbar^2 &= \langle j,m | J^2 - J_z^2 \mp \hbar J_z | j,m \rangle \\
    &= [j(j+1) - m^2 \mp m] \hbar^2 \braket{j,m}{j,m} \\
    &= (j\mp m)(j\pm m + 1) \hbar^2
\end{split}\\
C_\pm (j,m) = \sqrt{(j\mp m)(j\pm m + 1)}
\end{gather}
故
\begin{equation}
J_\pm \ket{j,m} = \sqrt{(j\mp m)(j\pm m + 1)} \hbar \ket{j,m \pm 1}
\end{equation}
矩阵元为
\begin{gather}
\langle j', m' | J_+ | j,m \rangle = \sqrt{(j-m)(j+m+1)} \hbar \delta_{j,j'} \delta_{m,m'} \\
\langle j', m' | J_- | j,m \rangle = \sqrt{(j+m)(j-m+1)} \hbar \delta_{j,j'} \delta_{m,m'}
\end{gather}
\end{frame}

\begin{frame}{$J_x$和$J_y$的矩阵元}
\begin{gather}
\begin{split}
J_x \ket{j,m} &= \frac{1}{2}(J_+ + J_-) \ket{j,m} \\
    &= \frac{1}{2} \sqrt{(j-m)(j+m+1)}\hbar \ket{j,m+1} \\
    & \quad +\frac{1}{2} \sqrt{(j+m)(j-m+1)}\hbar \ket{j,m-1}
\end{split} \\
\begin{split}
J_y \ket{j,m} &= \frac{1}{2i}(J_+ - J_-) \ket{j,m} \\
    &= \frac{1}{2i} \sqrt{(j-m)(j+m+1)}\hbar \ket{j,m+1} \\
    & \quad -\frac{1}{2i} \sqrt{(j+m)(j-m+1)}\hbar \ket{j,m-1}
\end{split}
\end{gather}
\end{frame}

\begin{frame}{$J_x$和$J_y$的矩阵元}
\begin{gather}
\begin{split}
\bra{j',m'} J_x \ket{j,m} &= \frac{1}{2} \sqrt{(j-m)(j+m+1)}\hbar \delta_{j,j'} \delta_{m+1,m'} \\
    & \quad +\frac{1}{2} \sqrt{(j+m)(j-m+1)}\hbar \delta_{j,j'} \delta_{m-1,m'}
\end{split} \\
\begin{split}
\bra{j',m'} J_y \ket{j,m} &= \frac{1}{2i} \sqrt{(j-m)(j+m+1)}\hbar \delta_{j,j'} \delta_{m+1,m'} \\
    & \quad -\frac{1}{2i} \sqrt{(j+m)(j-m+1)}\hbar \delta_{j,j'} \delta_{m-1,m'}
\end{split}
\end{gather}
\end{frame}


\end{document}
